\documentclass[12pt,a4paper,openright,oneside]{memoir}

\usepackage{iftex}
\ifXeTeX
\usepackage{fontspec}
\defaultfontfeatures{Ligatures=TeX}
\setmainfont{EB Garamond}[
  Scale = 1.0
]
\else
\usepackage[T1]{fontenc}
\usepackage[utf8]{inputenc}
\usepackage{mathptmx}
\fi

\usepackage{polyglossia}
\setdefaultlanguage{portuguese}
\setotherlanguage{english}

\usepackage[a4paper,top=3cm,bottom=3cm,left=2cm,right=2cm]{geometry}

\OnehalfSpacing

\maxsecnumdepth{subsection}
\setcounter{secnumdepth}{3}
\setsecnumformat{\csname the#1\endcsname\quad}

\chapterstyle{section}
\renewcommand*{\chapnamefont}{\normalfont\Large\scshape}
\renewcommand*{\chaptitlefont}{\normalfont\Huge\bfseries}

\usepackage{caption}
\DeclareCaptionLabelFormat{ipbeja}{#1~#2}
\captionsetup{
  labelfont=bf,
  labelsep=none,
  format=plain,
  textfont=it,
  justification=justified,
  singlelinecheck=false,
  labelformat=ipbeja
}

\captionsetup[table]{position=top}

\usepackage{graphicx}
\usepackage{float}
\usepackage{booktabs}
\usepackage{longtable}
\usepackage{amsmath}
\usepackage{xcolor}
\usepackage{fancyvrb}
\usepackage{framed}
\usepackage{calc}
\usepackage{etoolbox}

\usepackage{minted}
\setminted{
  linenos,
  breaklines,
  frame=lines,
  fontsize=\small
}

\usepackage[style=apa, backend=biber]{biblatex}
\addbibresource{bibliography.bib}
\usepackage{csquotes}

\newcommand{\Institute}{Instituto Politécnico de Beja}
\newcommand{\School}{Escola Superior de Tecnologia e Gestão}
\newcommand{\Degree}{Licenciatura em Engenharia Informática}
\newcommand{\Course}{Sistemas de Informação}
\newcommand{\Title}{Exploração e Desigualdade Laboral Global através
de Dados Abertos}
\newcommand{\Subtitle}{Engenharia de Dados}
\newcommand{\Author}{João Augusto Costa Branco Marado Torres}
\newcommand{\Advisor}{Dr.ª Isabel Sofia Sousa Brito}
\newcommand{\JuryMemberFirst}{Dr. João Paulo Trindade}
\newcommand{\JuryMemberSecond}{Dr.ª Elsa da Piedade Chinita Soares Rodrigues}
\newcommand{\Date}{Beja, dezembro de 2025}

\usepackage[hidelinks]{hyperref}
\usepackage{hyperxmp}
\hypersetup{
  pdfauthor={\Author},
  pdftitle={\Title},
  pdflicenseurl={https://creativecommons.org/licenses/by-sa/4.0/},
  pdfcopyright={© 2025 \Author --- CC BY-SA 4.0 for PDF, AGPL v3 for source},
}

\begin{document}

\thispagestyle{empty}

\begin{center}
  \includegraphics{Logotipo_IPBeja_horizontal-5/IPbeja_horizontal}

  \bigskip

  \textsc{\large \School}\\{\large \Degree}

  \bigskip

  \textsc{\large \Course}

  \vspace{4\baselineskip}

  \textsc{\Huge \Title}

  \smallskip

  {\Large \Subtitle}

  \bigskip

  {\large\bfseries \Author}

  \vfill

  \begin{center}
    \includegraphics[height=25mm,keepaspectratio]{Logotipos_ESTIG/estig}%
  \end{center}

  \vfill

  {\footnotesize \Date}
\end{center}

\cleardoublepage

\thispagestyle{empty}
\begin{center}
  \textsc{\large \Institute}

  \bigskip

  \textsc{\large \School}\\{\large \Degree}

  \bigskip

  \textsc{\large \Course}

  \vspace{4\baselineskip}

  \textsc{\Huge \Title}

  \smallskip

  {\Large \Subtitle}

  \bigskip

  {\large\bfseries \Author}

  \vspace{2\baselineskip}

  {\large Trabalho realizado no âmbito da unidade curricular de \Course}

  \vspace{2\baselineskip}

  \textsc{Orientação}

  \bigskip

  \Advisor

  \vfill

  {\footnotesize \Date}
\end{center}

\cleardoublepage

\thispagestyle{empty}
\begin{center}
  \textbf{Júri}

  \bigskip

  Responsável: \Advisor\\
  Vogal: \JuryMemberFirst\\
  Vogal: \JuryMemberSecond\\
\end{center}
\clearpage

\pagenumbering{roman}
% \fancyfoot[R]{\fontsize{8}{9}\selectfont\thepage}

% \chapter*{Resumo}
% \addcontentsline{toc}{chapter}{Resumo}
% \noindent
% ...
%
% \bigskip
%
% \textbf{Palavras-chave:} ...
%
% \chapter*{Abstract}
% \addcontentsline{toc}{chapter}{Abstract}
% \noindent
% ...
%
% \bigskip
%
% \textbf{Keywords:} ...
%
% \chapter*{Dedicatória}
% \addcontentsline{toc}{chapter}{Dedicatória}
% \begin{flushright}
%     ...
% \end{flushright}
%
% \chapter*{Agradecimentos}
% \addcontentsline{toc}{chapter}{Agradecimentos}
% ...

\clearpage
\tableofcontents
\clearpage
\listoffigures
\clearpage
\listoftables
\clearpage

\chapter*{Lista de Abreviaturas e Siglas}
\addcontentsline{toc}{chapter}{Lista de Abreviaturas e Siglas}
\begin{description}
  \item[ETL] \textit{Extract, Transform, Load}
  \item[FAIR] \textit{Findable, Accessible, Interoperable, Reusable}
  \item[FLOSS] \textit{Free \textit{Libre} and Open Source Software}
  \item[FMI] Fundo Monetário Internacional
  \item[GUI] \textit{Graphical User Interface}
  \item[IED] Investimento Estrangeiro Direto
  \item[ILO] \textit{International Labour Organization}
  \item[OLAP] \textit{Online Analytical Processing}
  \item[ONU] Organização das Nações Unidas
  \item[PIB] Produto Interno Bruto
  \item[POSIX] \textit{Portable Operating System Interface}
  \item[PPC] Paridade do Poder de Compra
  \item[URI] \textit{Uniform Resource Identifier}
\end{description}

% \chapter*{Simbologia e Notação}
% \addcontentsline{toc}{chapter}{Simbologia e Notação}
% \begin{description}
%   \item[$x$] variável independente
%   \item[$y$] variável dependente
% \end{description}

\clearpage
\pagenumbering{arabic}
% \fancyfoot[R]{\fontsize{8}{9}\selectfont\thepage}

\chapter{Introdução}
\label{ch:intro}

As transformações recentes do capitalismo por conta da crise da ordem
têm sido acompanhadas por um agravamento das desigualdades económicas
e sociais, particularmente no domínio do trabalho e da distribuição
do rendimento à escala mundial, levando à austeridade e reformas
laborais. Diferenças persistentes nos níveis salariais e na
estabilidade do emprego evidenciam assimetrias estruturais entre
países e regiões, onde pessoas vivem bem numa parte do mundo em troca
de outras viverem na miséria.

Este é o tema do trabalho: exploração global do trabalho e a
desigualdade num nível global.

A disponibilização de \textit{datasets} por múltiplos organismos
internacionais e instituições públicas cria possibilidades para a
análise destas desigualdades. Em particular, dados ao longo do tempo
(\textit{time series}), usando engenharia de dados e análise,
permitem observar a evolução histórica de indicadores laborais e
identificar tendências de longo prazo. Exemplos de tais indicadores
podem ser as participações do rendimento do trabalho no PIB Produto
Interno Bruto (PIB), coeficiente de Gini… (ver
Apêndice~\ref{ap:indicators} para definições e fórmulas). Era
interessante conseguir analisar como a divisão do trabalho na
dinâmica imperialista e neoliberal atual se manifestam nas
acumulações desiguais, nas dependências imperialistas, na
precariedade, nos salários, nas condições de trabalho…

Apesar da abundância de dados estatísticos disponíveis, a análise da
desigualdade laboral global permanece fragmentada. Os dados
encontram-se dispersos por múltiplas fontes e são frequentemente
analisados de forma isolada, através de indicadores únicos.

Esta separação dificulta a compreensão das relações entre salários,
rendimento, condições de trabalho e ação direta laboral (como as
greves), bem como a identificação de contradições estruturais do capitalismo.

\section{Objetivos}

O objetivo geral deste projeto é investigar a relação entre a
produtividade e parcelas salariais. Avaliar a dependência das trocas
comerciais e do Investimento Estrangeiro Direto (IED) no
\textit{labor share}. Analisar as principais razões para as greves,
por exemplo, se o salário não sobe por muito tempo. É apresentar as
contradições do capitalismo usando dados \textit{timeseries} abertos
num formato multidimensional e ferramentas livres demonstrando a
viabilidade técnica e ética dos mesmos. Pretende-se construir um
modelo de dados que permita observar, de forma conjunta, diferentes
indicadores relacionados com salários, rendimento e condições
laborais, entre países ou regiões e ao longo do tempo.

De forma mais específica, o projeto visa:
\begin{itemize}
  \item recolher e integrar dados provenientes de múltiplas fontes abertas;
  \item estruturar esses dados num modelo multidimensional adequado à análise;
  \item produzir análises e visualizações que evidenciem padrões,
    assimetrias e tendências relevantes.
\end{itemize}

\subsection{Questões de investigação}
\label{sec:questions}

Existe uma relação sistemática entre produtividade do trabalho e \textit{labor
share}?

Essa relação difere entre países centrais e periféricos?

Indicadores de dependência externa (IED) estão associados a menor
\textit{labor
share}?

Existe associação entre precariedade e ação direta?

\subsubsection{Hipóteses}

% Os trabalhadores trabalhando "melhor" receberão pior.
Em contextos de aumento da produtividade do trabalho, a parcela
salarial do PIB tende a permanecer estagnada ou a diminuir.

% Alguns países "sugam" os recursos de outros.
A inserção subordinada na economia global está associada à
transferência sistemática de valor do trabalho de países periféricos
para países centrais.

% Países com maior produtividade do trabalho apresentam, em média, maior
% parcela salarial do PIB.
%
% Países com aumento de produtividade do trabalho não apresentam aumento
% proporcional da parcela salarial do PIB.
%
% A relação entre produtividade e parcela salarial é mais positiva em países
% centrais do que em países periféricos.
%
% Quanto maior a dependência externa de um país (IED elevado e maior abertura
% comercial), menor é a parcela salarial do PIB.
%
% Períodos de estagnação ou queda da parcela salarial estão associados a maior
% número de greves ou conflitos laborais.

\section{Abordagem e Estrutura do relatório}

Para alcançar estes objetivos, o trabalho adota uma abordagem baseada
em engenharia de dados e o modelo de análise \textit{Online
Analytical Processing} (OLAP). O processo inclui a aplicação do
padrão \textit{Extract, Transform, Load} (ETL), a construção de um
\textit{data warehouse} e a realização de análises multidimensionais
via "cubos" (uma base de dados multidimensional em conceito), tabelas
e visualizações.

O relatório encontra-se organizado em três capítulos principais. Após
esta introdução, o capítulo de desenvolvimento apresenta o
enquadramento teórico, a metodologia adotada, a análise dos dados e a
discussão dos resultados. Por fim, o capítulo de conclusão sintetiza
os principais resultados, discute as limitações do estudo e aponta
possíveis direções para trabalhos futuros.

\chapter{Desenvolvimento}
\label{ch:dev}

\section{Enquadramento teórico}

Enquadramento metodológico geral

\subsection{Economia política}

\subsection{Desigualdade global}

\subsection{Divisão internacional do trabalho}

\subsection{Limitações dos dados estatísticos}
\label{sec:limits}

Com informações como estas, como se mede exatamente a exploração, a
dependência do imperialismo, a precariedade, ou a extração de valor.
Não é bem o objetivo medir a exploração diretamente de dados
monetários como o PIB e os outros. Talvez isso nem seja possível e
não é só uma questão de eu não o saber fazer. O meu objetivo é pegar
nesses dados e usá-los para fazer uma análise empírica, quer dizer,
observando os dados, e olhando para o mundo em que vivemos, e chegar
à conclusão que a minha análise bate certo com a realidade, e é
consistente com teorias da economia política. Eu penso que os dados
que escolho usar têm uma relação suficientemente forte para conseguir
testar a minhas hipóteses, apesar das limitações. Seria parecido às
formas de como é medida a qualidade de vida, que não é algo que se
meça diretamente.

Outra coisa é que pelos dados serem de fontes diferentes, podem usar
definições e metodologias diferentes para calcular o mesmo
(teoricamente), que depois pode causar-me problemas se eu não tiver
cuidado. É algo com que preciso de ter atenção na
etapa~\ref{sec:transform} do ETL.

\section{Metodologia}

\subsection{Escolha das técnologias e \textit{software}, e técnicas
de análise}

O meu foco é em criar algo que é facilmente reproduzível, sem ter que
te dar indicações de onde tens que clicar na interface gráfica para
elaborar uma ação específica. Por isso, decidi usar R ou Python para
o ETL, com ajuda de \textit{scripts} \textit{Portable Operating
System Interface} (POSIX) e \texttt{Makefile}s.

No início, ainda pensei em usar um simples \textit{script} POSIX. Mas
depois achei que ia se transformar em algo muito grande, complicado,
e com código repetido. Passei logo para a ideia de usar
\texttt{Makefile}s. Eles vão-me permitir orquestrar todos os passos a
executar neste trabalho, e caso eu mude algo no \textit{transform},
por exemplo, consigo pedir para executar tudo, mas na realidade ele
vai dar passar o passo da extração já que ele não foi afetado por
nenhuma mudança, então poupo um pouco de tempo.

Para o \textit{load}, usamos a base de dados DuckDB, que na
realidade, ela permite fazer o ETL inteiro através do que ela oferece
fora da caixa, e se necessário, extensões. Por exemplo, conseguimos
criar tabelas a partir de documentos usando o protocolo HTTP, por
exemplo, usar o \textit{statement} \mintinline{sql}{UNPIVOT} se
necessário, e no final, usar a operação de conjuntos
\mintinline{sql}{UNION [ALL] BY NAME} numa \textit{query}. Mas nós
apenas a vamos usar pelo facto de ela ter sido feita para conseguir
fazer operações OLAP de raiz, rapidamente. Com certeza uma
experiência para os tempos livres.

\begin{minted}[linenos, frame=lines, fontsize=\small]{sql}
CREATE OR REPLACE TABLE flights AS
    FROM 'https://duckdb.org/data/flights.csv';

UNPIVOT monthly_sales
ON COLUMNS(* EXCLUDE (empid, dept))
INTO
    NAME month
    VALUE sales;

SELECT * FROM capitals
UNION BY NAME
SELECT * FROM weather;
\end{minted}

Para a parte de análise, Jupyter
Notebooks ou R Markdown, ou apenas Python ou R com bibliotecas como Gnuplot ou
parecido. Depois consigo gerar PDFs e PostScripts.

No final, se tiver tempo, experimentar codificar os dados para um
formato de 4 estrelas no mínimo, RDF ou JSON-LD.

Penso em usar os serviços da Wikimedia Foundation também, caso
precise de informação extra, ou se eu quiser contribuir para o
projeto com a informação que conseguir retirar da análise.

Durante a análise, vou experimentar várias formas de mostrar as
informações: tabelas a partir do \textit{pivot}, correlações, regressões…

Quero depois disponibilizar os dados tratados em formatos abertos,
usando \textit{Uniform Resource Identifier} (URI) sempre que possível
para identificadores de países ou anos. Garantir que está tudo FAIR.
E documentar neste relatório o processo ETL.

Estas decisões não foram feitas ao calhas. Eu tenho uma ideia muito
forte de como todos devemos usar e criar software, incluindo, mas
especialmente para trabalhos académicos. A explicação para isso está
no Apêndice~\ref{ch:y_libre_n_open}.

\subsection{Estrutura do projeto}

Terá um ficheiro \texttt{README.md} que servirá como uma introdução
para quem pela primeira vez olha para este projeto, sem nenhum contexto.

Na pasta \texttt{data} vi estar guardado tudo o que são dados. Teria
uma pasta \texttt{raw} com os dados acabados de serem retirados da
\textit{source} e uma \texttt{cleaned} que guardasse as
transformações dos anteriores com todas as mudanças necessárias para
o seu uso na análise.

O que vai fazer as tarefas mencionadas em cima vão ser uma série de
\textit{scripts} localizados na pasta \texttt{cleaned} com executáveis.

O código para a criação do relatório e quase tudo relacionado à
documentação estaria na pasta \texttt{docs}.

Como a minha ideia é usar Python ou R, faz sentido que eu acabe por
usar Jupyter Notebooks ou R Markdown para complementar a
documentação, isso estará na pasta \texttt{notebooks}.

Código SQL fica na pasta \texttt{sql}.

\subsection{Fontes de dados}
\label{sec:sources}

Durante a minha pesquisa achei múltiplas potenciais fontes de dados
na Internet capazes de me ajudar nesta tarefa:

\begin{description}
  \item[Organização Internacional do Trabalho] Disponibiliza dados
    sobre empregabilidade, salários, \textit{labor share},
    sindicalização, horas de trabalho;
  \item[Banco Mundial] Para dados como o PIB, Gini, abertura
    comercial, fluxos de capital;
  \item[Our World In Data] Oferece diversas informações como a
    desigualdade, salário mínimo, ou sobre o trabalho informal e desemprego;
  \item[Wikidata] Para qualquer metadados que eu necessite, sobre
    países, empresas, sindicatos e protestos relacionados ao trabalho;
  \item[Confederação Sindical Internacional] Eles fazem os
    "\textit{ITUC Global Rights Index}" cada ano onde eu consigo
    encontrar informações como violações ao direito dos trabalhadores;
  \item[Base de dados sobre Desigualdade Mundial] Uma base de dados
    sobre a evolução histórica da distribuição mundial do rendimento
    e da riqueza, dentro como entre países.
\end{description}

Dado o caráter dos dados das fontes utilizadas, sem nenhuma relação
inicial, a análise baseia-se em indicadores indiretos, entendidos
como aproximações empíricas a processos sociais mais complexos, ou
seja, vamos relacionar esses valores para termos uma aproximação dos
valores que realmente queremos, e que na realidade, são impossíveis
de medir: exploração. Como foi falado na \ref{sec:limits}

Por exemplo, o PIB representa o valor de riqueza produzido num espaço
de tempo num determinado local. Com dados sobre o número de horas
trabalhadas durante esse espaço de tempo, eu consigo ter a razão do
PIB por hora de trabalho. Quanto maior esse valor, então mais riqueza
foi gerada em uma hora, maior a produtividade. Depois temos dados
como o \textit{labor share}: qual é a percentagem do rendimento
gerado num espaço de tempo num determinado local — pode ser a
percentagem do PIB — que foi para quem gerou o rendimento
(trabalhadores), por exemplo, em salários, ao invés de ir para o
capital. Com isto, posso fazer uma hipótese empírica de que, em
economias periféricas, aumentos de produtividade não se traduzem em
aumentos proporcionais da parcela salarial.

\subsection{\textit{Extract, Transform, and Load}}

\subsubsection{\textit{Extract}}

Como pôde ser visto em~\ref{sec:sources}, eu mencionei várias fontes
de dados, mas pensei que usar todas elas de uma só vez iria
complicar-me a vida muito. Então, para começar, escolhi usar apenas
duas: a Organização Internacional do Trabalho — vou começar a
mencionar por International Labour Organization (ILO) — e o Banco
Mundial para começar. A extração tem que ser feita de modo que no
futuro seja simples adicionar outras fontes de dados.

Para o ILO, decidi extrair informações como o \textit{labour income
share as a percent of GDP (\%)}, a quantidade de contratos de emprego
(formais e informais) divididos por sexo, atividade económica e setor
(público ou privado) do trabalhador, e a média do salário por hora
também dividido por sexo e atividade económica do trabalhador. Do
Banco Mundial, de cada país, extraí o PIB (no valor atual do dólar
dos Estados Unidos da América), o \textit{Gini index}, e as entradas
líquidas de IED (registado na balança de pagamentos e expresso em dólares).

Comecei pelo ILO. Descobri que eles oferecem uma biblioteca R —
\autocite{Rilostat} — para facilitar o meu trabalho. Ficou decidido
logo aqui que eu iria usar R ao invés de Python (apesar de que eu já
estava inclinado para usar o R). Aprendi o básico de R a partir do
\autocite{r_intro}. Depois, vi como podia usar a biblioteca que
encontrei, e espreitei as bibliotecas listadas nos \textit{imports} e
\textit{suggests} desta para saber o que faziam, já que poderiam
dar-me jeito — e deram. Qualquer dúvida, recorri ao \autocite{perplexity-ai}.

Quero mencionar que bem no início era para eu usar o
\autocite{openrefine}, por ser o que me parecia o mais semelhante ao
Power Query. Ele permite o \textit{extract} e \textit{transform} dos
dados, e essas ações, ficam guardadas num histórico, num ficheiro
JSON, que serviria para reproduzir as operações entre computadores.
Mas isso requisita usar a \textit{Graphical User Interface} (GUI)
mesmo assim; já não achei reproduzível o suficiente. Entretanto,
passei também os olhos na possibilidade de usar
\autocite{apache-airflow}, \autocite{luigi} e \autocite{prefect}.
Ficam aqui as menções.

Criei o ficheiro \texttt{extract.R} que apenas fazia o
\textit{download} dos \textit{datasets}.

Passando para o Banco Mundial, achei que seria mais simples apenas
usar \autocite{wget}. Renomeei o ficheiro anterior para
\texttt{extract\_ilostat.R} e criei o \texttt{extract\_worldbank.sh}.
Descobri logo que o \textit{link} que eles disponibilizavam na API
para fazer o \textit{download} do \textit{dataset} em \textit{comma
separated values} na realidade era um arquivo comprimido, mas lá dei
a volta a esse problema. Com o ficheiro com os dados, vem um com
metadados para cada código de país, e um com uma descrição sobre a
informação que eu tinha carregado da Internet.

Comecei a criar o \texttt{Makefile}. Criei uma regra que executa
todos os ficheiros na pasta \texttt{bin} que comecem com
\texttt{extract\_}. Para não ter que estar sempre a editar o ficheiro
e os \texttt{targets} com uma mudança na extração, usei a criação de
um ficheiro sem nada escrito, mas que a presença dele indica que o E
do ETL já foi concluído, e que depois pode ser usado para dependências.

\subsubsection{\textit{Transform}}
\label{sec:transform}

Usei o mesmo formato de nomear os ficheiros, e a mesma ideia no
\texttt{Makefile}, mas agora a regra \mintinline{make}{transform} depende do
\mintinline{make}{extract}.

Usei algumas das sugestões de bibliotecas que encontrei
anteriormente, nomeadamente o \autocite{dplyr} para facilitar a
manipulação dos \textit{data frames} e \autocite{stringr} para
manipular \textit{strings}. Também encontrei o \autocite{tidyr} que
me deixa usar colunas dos \textit{data frames} como variáveis,
facilitando o processo todo.

Nos dados do ILO queria renomear as colunas em todos os ficheiros
para o mesmo, e remover a linha que podia ser calculada a partir das
outras (o total do setor económico).

Para o Banco Mundial, quis renomear os ficheiros para algo menor que
explicasse o que estava presente neles. Tive que fazer um \textit{unpivot}
criando a coluna \mintinline{r}{year}. Mudei o nome de algumas colunas também
para ficar igual ao do ILO e juntei alguns dos metadados que tinham vindo
juntos.

Foi inicialmente adotada uma estratégia de consolidação dos
indicadores num único conjunto de dados antes do carregamento. Criei
um único ficheiro — \texttt{integrate.R} — onde eu leio cada ficheiro
de dados, e uso o código do país e no ano de cada linha para criar
linhas que agregam todos os valores correspondentes a esses dois que
os ficheiros separadamente possuíam. No \textit{data frame} do
salário por hora, tive que criar 3 novas colunas via \textit{pivot},
um para cada tipo de representação do salário, e depois escolhi usar
apenas o método Paridade do Poder de Compra (PPC), que me pareceu a
melhor opção comparado com o valor em dólar ou em moeda social.

No entanto, esta abordagem revelou-se inadequada do ponto de vista de
modelação multidimensional. Uma tabela não normalizada não é capaz de
se tornar numa \textit{data warehouse}. Estaríamos a adicionar uma
etapa a mais ao processo ETL que não existem e é conceitualmente
errado. Não existe tabelas de factos, nem de dimensões. Não existem
chaves primárias nem estrangeiras. E também não existe um
\textit{star/snowflake schema}.

Desnormalização e agregação são tarefas que só vão ser necessárias
mais tarde durante a análise.

Sem uma \textit{data warehouse}, não temos modelo multidimensional,
não conseguimos fazer OLAP, e torna-se complicado fazer análises.

A tarefa aqui no \textit{transform} é preparar os dados para a
criação da \textit{data warehouse} na próxima etapa \textit{load}.
Então para isso já temos que ter um modelo multidimensional em mente,
o qual foi o que faltou anteriormente, e por isso criamos uma etapa
que estraga o propósito do trabalho. O nosso modelo multidimensional
é detalhadamente explicado na \ref{sec:multidimensional}.

Sempre a atualizar o \texttt{Makefile}.

Neste momento existiam muitas linhas cheias de \mintinline{r}{NA},
exatamente porque essa informação não existe. O que fazer? Por serem
tudo números, uma das opções preencher todos os espaços vazios com
$0$. Mas isso não faz sentido.

\begin{itemize}
  \item PIB $0$ significava que nenhum dinheiro foi gerado, impossível;
  \item Coeficiente Gini $0$ era dizer que não existiam desigualdades,
    impossível no capitalismo;
  \item Participação do rendimento do trabalho no PIB de $0$ é algum
    tipo de escravidão;
  \item Média do salário por hora ser $0$ também é exploração total
    do trabalhador.
\end{itemize}

IED $0$ até é possível existir, mas improvável — o fluxo de entrada e
saída tinha que ser o mesmo, anulando-se —, e a ausência do valor não
significa que ele não existiu, apenas que não foi calculado, ou sei lá.

Devo remover as linhas que têm pelo menos uma falta de valor então?
Não, porque isso normalmente acontece nos países do sul global, e
isso tiraria a sua representatividade no \textit{dataset}.

Existem muitas formas de tratar os valores em falta. Se falta dados
em anos que ficam entre outros anos onde temos dados, podíamos
interpolar esses valores com ajuda de \textit{machine learning}.
Neste caso, apenas deixamos as linhas da forma que elas estão, sem
apagar nada também, por forma a preservar representatividade dos
países periféricos. Na altura da análise, é decidido o que fazer em
cada ocasião com os dados em falta.

\subsubsection{\textit{Load}}

A decisão ficou entre \autocite{duckdb} e \autocite{timescaledb} para
PostgreSQL, que me lembrei de ambas as menções de vídeos separados do
canal de YouTube \autocite{fireship-youtube}.

Optei pelo DuckDB por ser mais minimalista, estilo SQLite, enquanto
para o outro eu ia precisar de um servidor PostgreSQL.

A cena do DuckDB é que foi desenhado para suportar OLAP tratando as
tabelas não como um conjunto de linhas, mas mais como um conjunto de
colunas, e por isso é rápido nas operações OLAP.

Para esta parte do processo ETL, apenas escrevi SQL que o DuckDB
entende, e o \textit{script} \texttt{load.sh} manda executar esse
código no DuckDB e cria um ficheiro de base de dados presistente.

\subsection{Modelo multidimensional}
\label{sec:multidimensional}

O modelo multidimensional proposto organiza os dados em torno de uma
tabela de factos centrada em indicadores económicos e laborais,
articulada com dimensões temporais, geográficas e estruturais.

Os indicadores do Banco Mundial foram modelados através de uma tabela
de factos única. Cada valor para cada um daqueles indicadores
representa esse valor numa área do nosso planeta, num determinado
ano. A granularidade destes factos então é o país (em modo geral), o
ano, e o indicador, permitindo a análise conjunta e comparativa de
múltiplos indicadores macroeconómicos ao longo do tempo, evitando a
mistura de métricas com diferentes unidades e propriedades de
agregação. A tabela de factos observa of indicadores macroeconómicos
para as suas dimensões: país observado; indicador (o que estamos a
medir); e a data (neste caso o ano).

A tabela de dimensão país, vai ter as colunas (excluindo a chave
primária) para: o código ISO; o nome do país; a região; e o grupo de
rendimentos.

Depois pode ter outros conforme a necessidade.

O grupo de rendimentos, não é sempre o mesmo conforme o tempo vai
passando. Mas nós vamos usar o rótulo atual desse país (o que vem nos
ficheiros retirados da Internet), e que depois conforme o tempo passa
e essa informação é atualizada, nós atualizamos o rótulo desse país.

A tabela de dimensão indicador, vai ter as colunas para: o código do
indicador; o nome; a origem da informação; e a unidade do valor.

Os dados do ILO não permite existir apenas uma tabela de factos
porque a granularidade e a tabela de dimensões é diferente em alguns
casos. Apresentam maior granularidade e heterogeneidade do que os do
Banco Mundial, exigindo a modelação de múltiplas tabelas de factos
com diferentes granularidades, mantendo dimensões conformadas sempre
que possível.

No caso do \textit{labor share}, dá para adicionar à tabela de factos
para o Banco Mundial, já que a granularidade é a mesma.

Para a tabela de factos sobre os salários, usamos os dados o salário
médio por hora. A granularidade usa o país e o ano — como os outros —
mas para além disso, também o sexo, a atividade/setor económico, e a
forma de representação monetária.

Por último temos informação sobre os contratos (ou falta deles) de
trabalho, informal ou formal. Para ambas as situações, a
granularidade é a mesma, com o país, mês, sexo, atividade/setor
económico, e setor institucional (público, privado).

Também existe uma tabela de dimensão acerca da qualidade da
informação, usada por todas as tabelas de facto com dados da ILO.

O modelo multidimensional resultante assume a forma de uma
constelação de esquemas estrela, com múltiplas tabelas de factos
correspondentes a diferentes processos estatísticos (indicadores
macroeconómicos, salários, emprego), partilhando dimensões
conformadas como país, tempo, atividade económica e fonte.

Acabamos com um esquema \textit{galaxy} (ou será constelação ou
floco de neve?).

O nosso \textit{dataset} como está agora, é incompleto, no sentido
que não serve para responder a todas as questões apresentadas na
introdução~\ref{sec:questions}. Mas o \textit{data warehouse}
permite com que o modelo seja expandido mais tarde (especialmente
com as informações sobre sindicalização e ação direta).

\subsubsection{Tabelas de factos e dimensões}

A tabela de fatos é composta por indicadores económicos com relação
ao trabalho. Teria como medidas o \textit{labor share}, as
\textit{real wages}, o Gini, a produtividade, fluxos de IED, e dias
de greve. As dimensões seriam o tempo (ano, década), país, região,
setor, tipo de trabalho (formal ou informal).

Para o "cubo", usaria essas dimensões, como o tempo e a geografia —
países e regiões, norte e sul global, classificações dos países
conforme o Fundo Monetário Internacional (FMI) e a Organização das
Nações Unidos (ONU) —, como forma de "divisão" de dados para análise.

\section{Análise}

Na minha pesquisa para encontrar a melhor substituição ao Power BI,
achei o seguinte:

\begin{itemize}
  \item Metabase;
  \item Apache Superset;
  \item Grafana;
\end{itemize}

Todos depois teriam o mesmo problema que achei com OpenRefine.

Decidi que esta seria a melhor altura para usar RMarkdown e
bibliotecas como o \mintinline{r}{ggplot2} que também tinha sido
recomendado. Mas como não tive tempo para aprender como usar
RMarkdown, usei apenas um RScript.

No futuro pretendo experimentar o Streamlit.

\subsection{OLAP}

\subsection{Resultados principais}

A fase de análise inicial baseou-se num conjunto de dados
desnormalizado; a reformulação subsequente implementou um \emph{data
warehouse} multidimensional que permitiu a análise OLAP e a agregação
teoricamente fundamentada.

A análise apresentada abaixo põe isso mesmo em prática.

\subsubsection{\texorpdfstring{Evolução global do \emph{labour
share}}{Evolução global do labour share}}\label{evoluuxe7uxe3o-global-do-labour-share}

\begin{longtable}[]{@{}
  >{\raggedleft\arraybackslash}p{(\linewidth - 14\tabcolsep) * \real{0.0347}}
  >{\raggedleft\arraybackslash}p{(\linewidth - 14\tabcolsep) * \real{0.1389}}
  >{\raggedleft\arraybackslash}p{(\linewidth - 14\tabcolsep) * \real{0.1528}}
  >{\raggedleft\arraybackslash}p{(\linewidth - 14\tabcolsep) * \real{0.1806}}
  >{\raggedleft\arraybackslash}p{(\linewidth - 14\tabcolsep) * \real{0.1875}}
  >{\raggedleft\arraybackslash}p{(\linewidth - 14\tabcolsep) * \real{0.0972}}
  >{\raggedleft\arraybackslash}p{(\linewidth - 14\tabcolsep) * \real{0.0764}}
  >{\raggedleft\arraybackslash}p{(\linewidth - 14\tabcolsep) * \real{0.1319}}@{}}
\caption{Visão geral do labour share por região (2004--2025, a cada 5
anos).}\tabularnewline
\toprule\noalign{}
\begin{minipage}[b]{\linewidth}\raggedleft
year
\end{minipage} & \begin{minipage}[b]{\linewidth}\raggedleft
East Asia \& Pacific
\end{minipage} & \begin{minipage}[b]{\linewidth}\raggedleft
Europe \& Central Asia
\end{minipage} & \begin{minipage}[b]{\linewidth}\raggedleft
Latin America \& Caribbean
\end{minipage} & \begin{minipage}[b]{\linewidth}\raggedleft
Middle East \& North Africa
\end{minipage} & \begin{minipage}[b]{\linewidth}\raggedleft
North America
\end{minipage} & \begin{minipage}[b]{\linewidth}\raggedleft
South Asia
\end{minipage} & \begin{minipage}[b]{\linewidth}\raggedleft
Sub-Saharan Africa
\end{minipage} \\
\midrule\noalign{}
\endfirsthead
\toprule\noalign{}
\begin{minipage}[b]{\linewidth}\raggedleft
year
\end{minipage} & \begin{minipage}[b]{\linewidth}\raggedleft
East Asia \& Pacific
\end{minipage} & \begin{minipage}[b]{\linewidth}\raggedleft
Europe \& Central Asia
\end{minipage} & \begin{minipage}[b]{\linewidth}\raggedleft
Latin America \& Caribbean
\end{minipage} & \begin{minipage}[b]{\linewidth}\raggedleft
Middle East \& North Africa
\end{minipage} & \begin{minipage}[b]{\linewidth}\raggedleft
North America
\end{minipage} & \begin{minipage}[b]{\linewidth}\raggedleft
South Asia
\end{minipage} & \begin{minipage}[b]{\linewidth}\raggedleft
Sub-Saharan Africa
\end{minipage} \\
\midrule\noalign{}
\endhead
\bottomrule\noalign{}
\endlastfoot
2005 & 46.50462 & 51.90579 & 49.01374 & 34.62096 & 60.1775 & 47.44467 &
40.85213 \\
2010 & 46.75659 & 52.33425 & 49.57748 & 35.97861 & 59.7805 & 47.64883 &
40.73968 \\
2015 & 47.33338 & 51.57227 & 50.22565 & 41.33578 & 60.1975 & 46.72233 &
41.89534 \\
2020 & 47.96679 & 53.50062 & 49.95961 & 42.23030 & 60.4700 & 47.87667 &
41.81283 \\
2025 & 46.35841 & 52.11966 & 48.34071 & 38.25619 & 58.0670 & 45.85650 &
41.82416 \\
\end{longtable}

Conseguimos ver como os ``países de velho mundo'' são os únicos com um
\emph{labour share} acima dos 50\%. A nível global, isso não se
verifica. Embora que consiguamos registar um declínio da parte da
América do Norte de \(61\%\) em 2004, para \(58\%\) em 2025. A maioria
das regiões mostra certa estabilidade no valor (o que não significa ser
algo bom), tirando a América do Norte (com o declínio já mencionado), e
o Médio Oriente e Norte da África. O Médio Oriente e Norte da África,
Para além de terem o pior \emph{labour share} registado em (quase) todos
os anos (\(34\%\)--\(42\%\)), é bastante desregulado. Começou a subir
perto de 2007--2008, quanto teve a
\href{https://pt.wikipedia.org/wiki/Crise_financeira_de_2007\%E2\%80\%932008}{crise
global}. São países com grandes riquezas petrolíferas, e que quando o
preço do ``ouro negro'' subiu, o mesmo não se verificou na carteira dos
trabalhadores. A
\href{https://pt.wikipedia.org/wiki/Primavera_\%C3\%81rabe}{Primavera
Árabe} ocorreu após essa crise, e é provável que, para acalmar as
revoltas, os governos criaram reformas que aumentassem o \emph{labour
share} de certa forma. Também é possível notar o aumento a nível global
no ano 2020, com o evento do ínicio pandemia do
\href{https://pt.wikipedia.org/wiki/Coronav\%C3\%ADrus_da_s\%C3\%ADndrome_respirat\%C3\%B3ria_aguda_grave_2}{SARS-CoV-2}.

\includegraphics[width=1\linewidth]{analysis_files/figure-latex/labour-share-plot-1}

Uma coisa é certa, com um \emph{labour share} a nível global de cerca de
\(48\%\), e regiões onde o valor é \(10\%\) acima ou abaixo desse valor,
com tendências descendentes nas regiões de ambos os casos. Existe
claramente uma desigualdade regional, com os trabalhadores do norte
global a ganhar uma maior parte da riqueza que a sua região gera. Apesar
de declínios a nível regional, existe uma certa estabilidade global.

\pandocbounded{\includegraphics[keepaspectratio]{analysis_files/figure-latex/labour-share-heatmap-1.pdf}}

O \emph{heatmap} claramente destaca os mínimos das regiões do Médio
Oriente e Norte da África/Subsaariana em comparação com os máximos da
América do Norte.

\subsubsection{Comparação entre centro e
periferia}\label{comparauxe7uxe3o-entre-centro-e-periferia}

Conseguimos ver claramente uma diferença enorme entre os países com
maior rendimento e aqueles com menor rendimento. Existe uma diferença
maior que \(10\%\) sobre o \emph{labour share}. Os países de renda médio
alta e de renda médio baixa estão quase na mesma situação, e acima da
média dos extremos normalmente, com os países de renda média baixa a
passar para baixo nos últimos anos. Conseguimos reparar que os de renda
média baixa já tiveram em média maior \emph{labour share} que os de
renda médio alta até 2012. Mas também é de lembrar que talvez, os países
que hoje estão categorizados como sendo de renda média alta, antes de
2010 era de renda média baixa, e vice-versa, como já foi notado
anteriormente.

\includegraphics[width=1\linewidth]{analysis_files/figure-latex/dependency-theory-1}

Uma possível forma de demonstrar a
\href{https://pt.wikipedia.org/wiki/Teoria_da_depend\%C3\%AAncia}{teoria
da dependência} mostrando diferenças persistentes no labour share entre
economias centrais (alta renda) e periféricas (baixa renda), destacando
a dinâmica da exploração.

Os países centrais de rendimentos altos mantêm cerca de \(50\%\) do
labour share, enquanto os países periféricos de baixo rendimento
estagnam em cerca de \(38\%\), criando uma diferença estrutural de um
pouco mais de \(10\%\) que visualiza o mecanismo de ``dependência'' e
extração central da teoria da dependência.

\subsubsection{\texorpdfstring{Produtividade \emph{versus} \emph{labour
share}}{Produtividade versus labour share}}\label{produtividade-versus-labour-share}

Agora, vamos tentar mostrar uma contradição entre estas duas coisas:

\begin{itemize}
\tightlist
\item
  Produtividade (produção por trabalhador) --- medida em PIB por
  trabalhador (paridade do poder de compra em dólares), reflete o
  aumento da capacidade técnica da mão de obra e os ganhos de eficiência
  dentro do capitalismo. Isto captura diretamente a produção por
  trabalhador, tornando-o ideal para revelar a contradição entre o
  aumento da produtividade e a estagnação do \emph{labour share}. Como
  se concentra apenas nos trabalhadores empregados, evita distorções de
  fatores demográficos, como taxas de fertilidade ou reformados. O
  ajuste à paridade do poder de compra (PPC) garante a comparabilidade
  entre as diferentes regiões do globo;
\item
  Labor income share --- que representa a parte da riqueza efetivamente
  capturada pelos trabalhadores.
\end{itemize}

A produtividade tende a aumentar devido à inovação, mas \emph{labour
share} muitas vezes estagna ou diminui --- o que significa que o capital
captura mais da mais-valia.

A análise seguinte deve ajudar a tornar explícito esta contradição: os
trabalhadores aumentam cada vez mais a produtividade, enquanto a sua
reivindicação coletiva sobre o lucro (representado pelo \emph{labour
share}) é mediada e limitada pelo controlo do capital sobre a produção.

\includegraphics[width=1\linewidth]{analysis_files/figure-latex/productivity-labour-share-1}

Na maioria das regiões (excluindo o Médio Oriente e o Norte de África
até a altura do COVID), a produtividade só têm aumentado a nível global.
Enquanto isso, temos o \emph{labour share} basicamente estagnado.

\pandocbounded{\includegraphics[keepaspectratio]{analysis_files/figure-latex/mena-outlier-1.pdf}}

Se a relação fosse negativa ou estável, tornava explícito: à medida que
a produtividade cresce, a reivindicação dos trabalhadores não cresce.

Porém, afinal vemos que os ganhos de produtividade traduzem-se
parcialmente no \emph{labour share}, \textbf{mas apenas a nível global}.
E isso fica bem claro quando olhamos para a exceção: a região do Médio
Oriente e Norte de África, onde os rendimentos do petróleo, por estarem
sob controlo autoritário dos líderes da região, são eles que ficam com o
mais da mais-valia. A produtividade têm aumentado também, porém os
trabalhadores recebem cerca de \(35\%\)--\(40\%\), enquanto o capita
acumula mais de 60\%. O que isto mostra talvez na realidade é que esta é
a única região em que os capitalistas não aumentam, nem que ligeiramente
os salários, para manter os trabalhadores calmos, ao mesmo nível que do
resto do mundo. São dois modelos de extração do lucro diferentes: a
gerida e a descarada.

\begin{longtable}[]{@{}
  >{\raggedright\arraybackslash}p{(\linewidth - 8\tabcolsep) * \real{0.3913}}
  >{\raggedleft\arraybackslash}p{(\linewidth - 8\tabcolsep) * \real{0.1594}}
  >{\raggedleft\arraybackslash}p{(\linewidth - 8\tabcolsep) * \real{0.1449}}
  >{\raggedleft\arraybackslash}p{(\linewidth - 8\tabcolsep) * \real{0.1449}}
  >{\raggedleft\arraybackslash}p{(\linewidth - 8\tabcolsep) * \real{0.1594}}@{}}
\caption{Visão geral do labour share por região (2004--2025, a cada 5
anos).}\tabularnewline
\toprule\noalign{}
\begin{minipage}[b]{\linewidth}\raggedright
region
\end{minipage} & \begin{minipage}[b]{\linewidth}\raggedleft
slope
\end{minipage} & \begin{minipage}[b]{\linewidth}\raggedleft
std.error
\end{minipage} & \begin{minipage}[b]{\linewidth}\raggedleft
p.value
\end{minipage} & \begin{minipage}[b]{\linewidth}\raggedleft
statistic
\end{minipage} \\
\midrule\noalign{}
\endfirsthead
\toprule\noalign{}
\begin{minipage}[b]{\linewidth}\raggedright
region
\end{minipage} & \begin{minipage}[b]{\linewidth}\raggedleft
slope
\end{minipage} & \begin{minipage}[b]{\linewidth}\raggedleft
std.error
\end{minipage} & \begin{minipage}[b]{\linewidth}\raggedleft
p.value
\end{minipage} & \begin{minipage}[b]{\linewidth}\raggedleft
statistic
\end{minipage} \\
\midrule\noalign{}
\endhead
\bottomrule\noalign{}
\endlastfoot
South Asia & -0.0001262 & 0.0000310 & 0.0006620 & -4.0639675 \\
Sub-Saharan Africa & 0.0003493 & 0.0001454 & 0.0267061 & 2.4017787 \\
East Asia \& Pacific & -0.0000148 & 0.0000271 & 0.5912831 &
-0.5462030 \\
Middle East \& North Africa & -0.0006366 & 0.0000808 & 0.0000002 &
-7.8815888 \\
North America & -0.0001202 & 0.0000219 & 0.0000264 & -5.4987752 \\
Latin America \& Caribbean & -0.0000871 & 0.0000758 & 0.2652746 &
-1.1478594 \\
Europe \& Central Asia & -0.0000191 & 0.0000225 & 0.4047215 &
-0.8521926 \\
\end{longtable}

Quando olhamos \textbf{a nível regional}, a contradição mostra-se
correta (apenas a África Subsaariana contraria a tendência). A
``inclinação positiva'' global era uma ilusão ótica ---
\href{https://pt.wikipedia.org/wiki/Paradoxo_de_Simpson}{paradoxo de
Simpson} --- resultante da média de todas as regiões, mas regionalmente,
a tendência do capital é clara. Ao olharmos para o cenário globalmente,
acabamos por cair na narrativa do capital, que ganhos de produtividade
equivale a ganhos salariais. A análise regional expõe a mentira --- o
\emph{labour share} está basicamente estagnado em todos os lugares.

\includegraphics[width=1\linewidth]{analysis_files/figure-latex/unmasking-contradiction-1}

A linha global demonstra a ilusão de ótica resultante da mistura de
regiões alta produtividade com com baixa produtividade. O declínio do
\emph{labour share} é uma característica estrutural do capitalismo em
todo o mundo.

Vamos normalizar a produtividade numa distribuição normal para uma
última visualização.

\pandocbounded{\includegraphics[keepaspectratio]{analysis_files/figure-latex/focus-on-labour-share-1.pdf}}

\subsubsection{Investimento direto estrangeiro versus labour
share}\label{investimento-direto-estrangeiro-versus-labour-share}

Como os fluxos do IDE podem assumir valores negativos, aplicamos uma
transformação logarítmica com sinal, preservando tanto a magnitude como
a direção dos fluxos de capital, ao mesmo tempo que reduzimos a
assimetria.

Sem o uso da desta transformação, neste caso com a função logaritmica,
estariamos a juntar fluxos enormes com fluxos mais pequenos. Desta forma
os valores maiores ficam comprimidos, diminuindo a variação, continuam a
mostrar a sua dimensão, ao mesmo tempo que diminuimos os possíveis
\emph{outliers} que poderiam dificultar a análise do coeficiente de
correlação Pearson.

A transformação logarítmica com sinal preserva a direção do fluxo, ao
mesmo tempo que controla os valores monstruosos.

A correlação de Pearson usa a covariância que mede a direção do
movimento conjunto deste caso da média dos fluxos do IDE com a média do
\emph{labour share}, ao longo do tempo, e divide pela multipicação do
desvio padrão dos dois. O desvio padrão mede o quão dispersos estão os
dados em torno da sua média. Se estiverem muito dispersos, o valor é
maior, e isso vai então diminuir o valor do coeficiente de correlação.
No final, o coeficiente é um valor normalizado entre \(-1\) e \(1\). Se
o valor fôr \(0\), não existe correlação. Se fôr \(1\), existe uma
correlação prefeita onde um valor cresce juntamente com o outro, que é o
caso oposto de se o coeficiente fôr \(-1\), também uma correlação
prefeita, mas um dos valor desce quando o outro sobe.

Para a esta correlação entre os fluxos IDE e o \emph{labour share}, o
coeficiente \(< 0\) significa ``mais extração, menos \emph{labour
share}'' --- sinal imperialista fundamental.

Esta visualização destaca correlações negativas no Sul Global (América
Latina, Caraíbas, Médio Oriente e Norte de África) como evidência da
destruição salarial através da extração imperialistas, enquanto
correlações positivas em outros lugares expõem a colaboração entre
burguesias.

\pandocbounded{\includegraphics[keepaspectratio]{analysis_files/figure-latex/correlation-1.pdf}}

As correlações negativas na América Latina (\(-4,1\%\)) e na região
Médio Oriente e Norte de África (\(-10,4\%\)) dão força à tese de que os
fluxos IDE não representam ``desenvolvimento'', mas sim a extração de
valor dos trabalhadores para os centros imperialistas. O Leste Asiático
e o Pacífico apresentam a correlação positiva mais forte entre fluxos
IDE e \emph{labour share} (Ásia Oriental \(+17,7\%\)), sugerindo
resistência dos trabalhadores ou dinâmicas alternativas de capital como
alianças capitalistas estatais. A correlação positiva da Ásia Oriental
prova que a autogestão dos trabalhadores pode capturar/reverter os
fluxos imperialistas.

Os valores do coeficiente de correlação, podem não parecer
suficientemente fortes, até as regiões destacadas têm valores fracos.
Mas esta é uma análise empírica. As fracas correlações não negam a
extração imperialista. Correlações fracas são consistentes com um
sistema no qual a extração de excedentes está estruturalmente
incorporada.


\section{Discussão}

Na União Europeia, por exemplo, para conseguirem ter mais dinheiro
para gastar em fins bélicos, vemos acontecer por vários países da
federação, cortes orçamentais na segurança social e na educação e
saúde pública, e com isso, reformas nas leis laborais com o intuito
de aumentar a produtividade, que em troca, traz o aumento da
precariedade do trabalhador e o aumento da desigualdade económica se
a parcela salarial não aumentar com o aumento da produtividade. Por
isso vemos na Europa várias greves gerais quase em simultâneo: 2 de
dezembro na França, 28 de novembro e 12 de dezembro na Itália, de 24
a 26 de novembro na Bélgica, e aqui em Portugal a 11 de dezembro. Os
sindicatos falham ao não relacionarem estes eventos, demonstrando que
o problema não desaparece quando passamos as bordas do país, e que a
luta e resistência é internacionalista, neste caso, é a nível europeu.

Na União Europeia, por exemplo, para conseguirem ter mais dinheiro
para gastar em fins bélicos, vemos acontecer por vários países da
federação, cortes orçamentais na segurança social e na educação e
saúde pública, e com isso, reformas nas leis laborais com o intuito
de aumentar a produtividade, que em troca, traz o aumento da
precariedade do trabalhador e o aumento da desigualdade económica se
a parcela salarial não aumentar com o aumento da produtividade. Por
isso vemos na Europa várias greves gerais quase em simultâneo.

\chapter{Conclusão}
\label{ch:conclusion}

\clearpage
\printbibliography[title={Referências Bibliográficas}]

\clearpage
\chapter*{Licença}
\addcontentsline{toc}{chapter}{Licença}
\noindent
Este documento está licenciado sob uma
\href{https://creativecommons.org/licenses/by-sa/4.0/}{Licença
  Creative Commons Atribuição–Partilha nos Mesmos Termos 4.0
Internacional (CC BY-SA 4.0)}.

\vspace{0.5cm}
O código fonte (ficheiros \texttt{.tex}, \texttt{.bib},
\texttt{Makefile}, etc.) utilizado para produzir este relatório está
licenciado sob a
\href{https://www.gnu.org/licenses/agpl-3.0.html}{GNU Affero General
Public License v3.0 (AGPL v3)}.

\clearpage
\appendix

\chapter{Indicadores}
\label{ap:indicators}

\section{PIB}

GDP

GDP (current US\$)

Gross domestic product is the total income earned through the production of
goods and services in an economic territory during an accounting
period. It can
be measured in three different ways: using either the expenditure
approach, the
income approach, or the production approach. This indicator is expressed in
current prices, meaning no adjustment has been made to account for price
changes over time. This indicator is expressed in United States dollars.

\section{Participação do rendimento do trabalho}

Labour (income) share

\section{Índice de Gini}

Gini index

Gini index measures the extent to which the distribution of income
(or, in some
cases, consumption expenditure) among individuals or households within an
economy deviates from a perfectly equal distribution. A Lorenz curve plots the
cumulative percentages of total income received against the cumulative number
of recipients, starting with the poorest individual or household. The Gini
index measures the area between the Lorenz curve and a hypothetical line of
absolute equality, expressed as a percentage of the maximum area under the
line. Thus a Gini index of 0 represents perfect equality, while an
index of 100
implies perfect inequality.

\section{Investimento direto estrangeiro}

Investimento direto estrangeiro, entradas líquidas (Balança de
Pagamentos, US\$ atuais)

Foreign direct investment, net inflows (BoP, current US\$)

Foreign direct investment refers to direct investment equity flows in the
reporting economy. It is the sum of equity capital, reinvestment of earnings,
and other capital. Direct investment is a category of cross-border investment
associated with a resident in one economy having control or a significant
degree of influence on the management of an enterprise that is resident in
another economy. Ownership of 10 percent or more of the ordinary shares of
voting stock is the criterion for determining the existence of a direct
investment relationship. Data are in current U.S. dollars.

\section{Profit sharing}

\section{Real wages}

\chapter{Justificação das escolhas tecnológicas}
\label{ch:y_libre_n_open}

Eu uso Linux e não Microsoft Windows já faz 5 anos. Por várias
razões, mas tudo
tem a mesma raiz do problema: Windows é \textit{software} proprietário.

\section{\textit{Software} livre}

Eu vou, sempre que conseguir, usar \textit{free, \textit{libre}, and open
source software} (FLOSS). Muitas vezes tenho que ceder aos monopólios do
\textit{software}, porque não o fazer prejudica-me na vida fora do
digital. Por
exemplo, parece que muita da infraestrutura informática do politécnico depende
da Microsoft, a começar pelos correios eletrónicos disponibilizados para os
alunos. Eu sendo aluno, tenho que o usar porque senão perco acesso
à informação
importante para completar a minha formação. E é assim que estes oligarcas
digitais criam dependências nos seus \textit{software}. Mas eles
são capazes de
se introduzir em muito mais do que apenas na vida académica e nos
estudos. Quer
dizer, caraças, eles metem-se em tudo o que conseguem, desde que haja lucro
sendo feito.

Na pedagogia e ensino, devia ser ilegal o uso de \textit{software}
proprietário, já que este vai contra todos os fundamentos da educação.
\textit{Software} livre tem que ser a norma. O \textit{software} que deixa o
estudante aprender e estudar o \textit{software} que usa, executar o
\textit{software} da forma que lhe for mais conveniente, modificar
para as suas
necessidades, e conseguir partilhar a sabedoria que adquiriu com qualquer um.

Acho que já deu para ter uma ideia do porquê da minha indignação. E para minha
surpresa, esta cadeira pede para o aluno usar Microsoft Excel, Microsoft Power
BI e Microsoft SQL Server. Primeiro, que eu nem consigo usar alguns desses
\textit{softwares} no meu portátil sem ter que ter o Windows como sistema
operativo — isso é que nem pensar. E depois que existem alternativas, e boas
(senão melhores) que são livres. Vou mencionar OpenRefine e Apache
Superset que
fazem tudo o que era necessário. Não os usei porque, igual às ferramentas da
Microsoft, esses não oferecem uma forma simples de reproduzir as mesmas ações
em qualquer outra máquina.

\section{Dados Abertos}

Falando mais sobre a cadeira, uns tópicos que eu acho que deviam
ser pelo menos
mencionados é o dos dados abertos mencionando o esquema da implementação das 5
estrelas e os princípios \textit{Findable, Accessible,
Interoperable, Reusable}
(FAIR). Mas claramente não querem saber disso. Incentivam a usar o formato
proprietário do Excel (duas estrelas). Não querem nos ensinar a explorar,
aprender, adquirir um pensamento crítico independente, mas sim a consumir a
droga do \textit{software} proprietário, e aceitar que tem que ser assim, que
estas são as ferramentas corretas.

Não querem ensinar como aprender, mas sim o que aprender. A resistência em
adotar a educação aberta decorre do desejo de controlar o que
aprendemos. E nem
tentam esconder mais isto. É triste perguntar a razão de estarmos a aprender
algo e a resposta ser “porque é o que as empresas querem”.

\backmatter

\end{document}

% CITATIONS TO CLAIMS ???

% CITE this repo
