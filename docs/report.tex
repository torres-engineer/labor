\documentclass[12pt,a4paper,openright,oneside]{memoir}

\usepackage{iftex}
\ifXeTeX
  \usepackage{fontspec}
  \defaultfontfeatures{Ligatures=TeX}
  \setmainfont{EB Garamond}[
    Scale = 1.0
  ]
\else
  \usepackage[T1]{fontenc}
  \usepackage[utf8]{inputenc}
  \usepackage{mathptmx}
\fi

\usepackage{polyglossia}
\setdefaultlanguage{portuguese}
\setotherlanguage{english}

\usepackage[a4paper,top=3cm,bottom=3cm,left=2cm,right=2cm]{geometry}

\OnehalfSpacing

\maxsecnumdepth{subsection}
\setcounter{secnumdepth}{3}
\setsecnumformat{\csname the#1\endcsname\quad}

\chapterstyle{section}
\renewcommand*{\chapnamefont}{\normalfont\Large\scshape}
\renewcommand*{\chaptitlefont}{\normalfont\Huge\bfseries}

\usepackage{caption}
\DeclareCaptionLabelFormat{ipbeja}{#1~#2}
\captionsetup{
  labelfont=bf,
  labelsep=none,
  format=plain,
  textfont=it,
  justification=justified,
  singlelinecheck=false,
  labelformat=ipbeja
}

\captionsetup[table]{position=top}

\usepackage{graphicx}
\usepackage{float}
\usepackage{booktabs}
\usepackage{longtable}

\usepackage{minted}
\setminted{
  linenos,
  breaklines,
  frame=lines,
  fontsize=\small
}

\usepackage[style=apa, backend=biber]{biblatex}
\addbibresource{bibliography.bib}
\usepackage{csquotes}

\newcommand{\Institute}{Instituto Politécnico de Beja}
\newcommand{\School}{Escola Superior de Tecnologia e Gestão}
\newcommand{\Degree}{Licenciatura em Engenharia Informática}
\newcommand{\Course}{Sistemas de Informação}
\newcommand{\Title}{Exploração e Desigualdade Laboral Global através de Dados Abertos}
\newcommand{\Subtitle}{Engenharia de Dados}
\newcommand{\Author}{João Augusto Costa Branco Marado Torres}
\newcommand{\Advisor}{Dr.ª Isabel Sofia Sousa Brito}
\newcommand{\JuryMemberFirst}{Dr. João Paulo Trindade}
\newcommand{\JuryMemberSecond}{Dr.ª Elsa da Piedade Chinita Soares Rodrigues}
\newcommand{\Date}{Beja, novembro de 2025}

\usepackage[hidelinks]{hyperref}
\usepackage{hyperxmp}
\hypersetup{
  pdfauthor={\Author},
  pdftitle={\Title},
  pdflicenseurl={https://creativecommons.org/licenses/by-sa/4.0/},
  pdfcopyright={© 2025 \Author --- CC BY-SA 4.0 for PDF, AGPL v3 for source},
}

\begin{document}

\thispagestyle{empty}

\begin{center}
    \includegraphics{Logotipo_IPBeja_horizontal-5/IPbeja_horizontal}

    \bigskip

    \textsc{\large \School}\\{\large \Degree}

    \bigskip

    \textsc{\large \Course}

    \vspace{4\baselineskip}

    \textsc{\Huge \Title}

    \smallskip

    {\Large \Subtitle}

    \bigskip

    {\large\bfseries \Author}

    \vfill

    \begin{center}
        \includegraphics[height=25mm,keepaspectratio]{Logotipos_ESTIG/estig}%
    \end{center}

    \vfill

    {\footnotesize \Date}
\end{center}

\cleardoublepage

\thispagestyle{empty}
\begin{center}
    \textsc{\large \Institute}

    \bigskip

    \textsc{\large \School}\\{\large \Degree}

    \bigskip

    \textsc{\large \Course}

    \vspace{4\baselineskip}

    \textsc{\Huge \Title}

    \smallskip

    {\Large \Subtitle}

    \bigskip

    {\large\bfseries \Author}

    \vspace{2\baselineskip}

    {\large Trabalho realizado no âmbito da unidade curricular de \Course}

    \vspace{2\baselineskip}

    \textsc{Orientação}

    \bigskip

    \Advisor

    \vfill

    {\footnotesize \Date}
\end{center}

\cleardoublepage

\thispagestyle{empty}
\begin{center}
  \textbf{Júri}

  \bigskip

  Responsável: \Advisor\\
  Vogal: \JuryMemberFirst\\
  Vogal: \JuryMemberSecond\\
\end{center}
\clearpage

\pagenumbering{roman}
% \fancyfoot[R]{\fontsize{8}{9}\selectfont\thepage}

% \chapter*{Resumo}
% \addcontentsline{toc}{chapter}{Resumo}
% \noindent
% ...
%
% \bigskip
%
% \textbf{Palavras-chave:} ...
%
% \chapter*{Abstract}
% \addcontentsline{toc}{chapter}{Abstract}
% \noindent
% ...
%
% \bigskip
%
% \textbf{Keywords:} ...
%
% \chapter*{Dedicatória}
% \addcontentsline{toc}{chapter}{Dedicatória}
% \begin{flushright}
%     ...
% \end{flushright}
%
% \chapter*{Agradecimentos}
% \addcontentsline{toc}{chapter}{Agradecimentos}
% ...

\clearpage
\tableofcontents
\clearpage
% \listoffigures
% \clearpage
% \listoftables
% \clearpage

% \chapter*{Lista de Abreviaturas e Siglas}
% \addcontentsline{toc}{chapter}{Lista de Abreviaturas e Siglas}
% \begin{description}
%     \item[a11y] \textit{Accessibility}
%     \item[API] \textit{Application Programming Interface}
%     \item[DX] \textit{Developer Experience}
%     \item[FLOSS] \textit{Free Libre and Open Source Software}
%     \item[HTTP] \textit{Hypertext Transfer Protocol}
%     \item[i18n] \textit{Internationalization}
%     \item[MVC] \textit{Model-View-Controller}
%     \item[POO] Programação Orientada a Objetos
%     \item[REST] \textit{Representational State Transfer}
%     \item[TDD] \textit{Test-Driven Development}
%     \item[UML] \textit{Unified Modeling Language}
% \end{description}

% \chapter*{Simbologia e Notação}
% \addcontentsline{toc}{chapter}{Simbologia e Notação}
% \begin{description}
%   \item[$x$] variável independente
%   \item[$y$] variável dependente
% \end{description}

\clearpage
\pagenumbering{arabic}
% \fancyfoot[R]{\fontsize{8}{9}\selectfont\thepage}

\chapter{Introdução}
\label{ch:intro}

Eu uso Linux e não Microsoft Windows já faz 5 anos. Por várias razões, mas tudo
tem a mesma raiz do problema: Windows é software proprietário.

Eu vou sempre que conseguir usar FLOSS. Muitas vezes tenho que ceder aos
monopólios do software, porque não o fazer prejudica-me na vida fora do
digital. Por exemplo, parece que muita da infraestrutura informática do
politécnico depende da Microsoft, a começar pelos correios eletrónicos
disponibilizados para os alunos, eu sendo aluno, tenho que o usar porque senão
perco acesso a informação importante para completar a minha formação. E é assim
que estes oligarcas digitais criam dependências. Mas eles são capazes de se
introduzir ainda mais na vida académica dos estudos. Quer dizer, caraças, eles
metem-se em tudo o que conseguem, seja para controlar ou ser controlado, desde
que haja lucro sendo feito.

Na pedagogia e ensino, devia ser ilegal o uso de software proprietário, já que
este vai contra todos os fundamentos da educação. FLOSS tem que ser a norma. O
software que deixa o estudante aprender e estudar o software que usa, executar
o software da forma que lhe for mais conveniente, modificar para as suas
necessidades, e conseguir partilhar a sabedoria que adquiriu com qualquer um.

Acho que já deu para ter uma ideia do porquê da minha indignação. E para minha
surpresa, está cadeira pede para o aluno usar Microsoft Excel, Microsoft Power
BI e Microsoft SQL Server. Primeiro que eu nem consigo usar alguns desses
softwares no meu portátil sem ter que ter o Windows, e é que nem pensar. E
depois que existem alternativas, e boas (senão melhores) que são FLOSS. Vou
mencionar OpenRefine e Apache Superset que fazem tudo o que era necessário. Não
os usei porque, igual as ferramentas da Microsoft, esses não oferecem uma forma
simples de reproduzir as mesmas ações em qualquer outra máquina.

Falando mais sobre a cadeira, uns tópicos que eu acho que deviam ser pelo menos
mencionados é o dos dados abertos mencionando o esquema da implementação das 5
estrelas e os princípios FAIR (Findable, Accessible, Interoperable, Reusable).
Mas claramente não querem saber disso. Incentivam a usar o formato proprietário
do Excel, duas estrelas. Não querem nos ensinar a explorar, aprender, adquirir
um pensamento crítico independente mas sim a consumir a droga do software
proprietário, e aceitar que tem que ser assim, que estas são as ferramentas
corretas.

Não querem ensinar como aprender mas sim o que aprender. A resistência em
adotar a educação aberta decorre do desejo de controlar o que aprendemos. E
nem tentam esconder mais isto. É triste perguntar a razão de estarmos a
aprender algo e a resposta ser “porque é o que as empresas querem”.

A minha tarefa inicial é pegar em várias séries temporais, e aplicando o padrão
ETL, acabar com um modelo de dados “cubo” (uma base de dados multidimensional
em conceito) que permita o modelo de análise OLAP. No final, fazer uma análise
da informação possível de obter através do data warehouse.

O tema que escolhi para o trabalho é relacionado à exploração global do
trabalho e a desigualdade num nível global, usando parcelas salariais do PIB,
coeficiente de Gini… e era interessante conseguir analisar como a divisão do
trabalho na dinâmica imperialista e neoliberal atual se manifestam nas
acumulações desiguais, dependências imperialistas, precariedade, salários,
condições de trabalho…

A minha ideia é provar que existe uma dialética entre a produtividade e
parcelas salariais, os trabalhadores trabalhando melhor receberão pior. Tentar
relacionar a dependência em trocas comerciais e o IED com a repartição do valor
agregado, mostrar como alguns países sugam os recursos de outros. Relacionar a
participação nos lucros e resultados com a repartição do valor agregado para
calcular o quão explorados estamos a ser numa determinada região do mundo.
Encontrar quais são as principais razões para as greves, por exemplo, se o
salário não sobe por muito tempo. É apresentar as contradições do capitalismo
usando dados abertos e ferramentas FLOSS demonstrando a viabilidade técnica e
ética dos mesmos.

Durante a minha pesquisa achei multipas potenciais fontes de dados na Internet
capazes de me ajudar nesta tarefa:

\begin{description}
    \item[Organização Internacional do Trabalho] Disponibiliza dados sobre empregabilidade, salários, parcelas salariais, sindicalização, horas de trabalho;
    \item[Banco Mundial] Para dados como o PIB, Gini, abertura comercial, fluxos de capital;
    \item[Our World In Data] Oferece diversas informções como a desigualdade, salário mínimo, ou sobre o trabalho informal e desemprego;
    \item[Wikidata] Para qualquer metadado que eu necessite, sobre países, empresas, sindicatos e protestos relacionados ao trabalho;
    \item[Confederação Sindical Internacional] Eles fazem os "ITUC Global Rights Index" cada ano onde eu consigo encontrar informações como violações ao direito dos trabalhadores;
    \item[Base de dados sobre Desigualdade Mundial] Uma base de dados sobre a evolução histórica da distribuição mundial do rendimento e da riqueza, dentro como entre países.
\end{description}

Para o "cubo", usaria o tempo (anos e décadas) como uma das dimenções, e a
geografia (países e regiões, norte e sul global, classificações dos países de
acordo com a FMI e a ONU) como outro. Depois talvez também dê para usar os
setores de trabalho e se o trabalho é formal ou informal como outra forma de
"divisão" de dados para análise. Depois claro informações sobre o trabalho
(salários, horas de trabalho, parcelas salariais) e sobre o capital (PIB per
capita, fluxos IED, participação nos lucros e resultados).

No futuro tenho que pensar melhor na elaboração de uma tabela de fatos.

Como queria fazer algo reproduzível, decidi usar R ou Python para o ETL, com
ajuda de scripts POSIX e Makefiles. Para o load eu achei a base de dados
DuckDB, que depois vim a descobrir que basicamente consigo fazer o ETL inteiro
com essa base de dados. Para a parte de análise, Jupyter Notebooks ou R
Markdown, ou apenas Python ou R com bibliotecas como Gnuplot ou parecido.
Depois consigo gerar PDFs e PostScripts.

No final, se tiver tempo, experimentar codificar os dados para um formato de 4
estrelas no mínimo, RDF ou JSON-LD.

Penso em usar os serviços da Wikimedia Foundation também, caso precise de
informação extra, ou se eu quiser contribuir para o projeto com a informação
que conseguir retirar da análise.

Durante a análise, vou experimentar várias formas de mostrar as informações:
tabelas a partir do \textit{pivot}, correlações, regreções…

Quero depois disponibilizar os dados tratados em formatos abertos, usando URIs
sempre que possível para identificadores de países ou anos. Garantir que está
tudo FAIR. E documentar neste relatório o processo ETL.

\chapter{Desenvolvimento}
\label{ch:dev}

\chapter{Conclusão}
\label{ch:conclusion}

\clearpage
\printbibliography[title={Referências Bibliográficas}]

\clearpage
\chapter*{Licença}
\addcontentsline{toc}{chapter}{Licença}
\noindent
Este documento está licenciado sob uma
\href{https://creativecommons.org/licenses/by-sa/4.0/}{Licença Creative Commons Atribuição–Partilha nos Mesmos Termos 4.0 Internacional (CC BY-SA 4.0)}.

\vspace{0.5cm}
O código fonte (ficheiros \texttt{.tex}, \texttt{.bib}, \texttt{Makefile}, etc.) utilizado para produzir este relatório está licenciado sob a
\href{https://www.gnu.org/licenses/agpl-3.0.html}{GNU Affero General Public License v3.0 (AGPL v3)}.

\clearpage
\appendix
% \chapter{Apêndice A: Questionário}

% \chapter{Apêndice: Linha de comandos}

\backmatter
% \chapter*{Anexos}
% \addcontentsline{toc}{chapter}{Anexos}
% \section*{Anexo A: Documento Externo}
% Texto de anexo.

\end{document}

