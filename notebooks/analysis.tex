A fase de análise inicial baseou-se num conjunto de dados
desnormalizado; a reformulação subsequente implementou um \emph{data
warehouse} multidimensional que permitiu a análise OLAP e a agregação
teoricamente fundamentada.

A análise apresentada abaixo põe isso mesmo em prática.

\subsubsection{\texorpdfstring{Evolução global do \emph{labour
share}}{Evolução global do labour share}}\label{evoluuxe7uxe3o-global-do-labour-share}

\begin{longtable}[]{@{}
  >{\raggedleft\arraybackslash}p{(\linewidth - 14\tabcolsep) * \real{0.0347}}
  >{\raggedleft\arraybackslash}p{(\linewidth - 14\tabcolsep) * \real{0.1389}}
  >{\raggedleft\arraybackslash}p{(\linewidth - 14\tabcolsep) * \real{0.1528}}
  >{\raggedleft\arraybackslash}p{(\linewidth - 14\tabcolsep) * \real{0.1806}}
  >{\raggedleft\arraybackslash}p{(\linewidth - 14\tabcolsep) * \real{0.1875}}
  >{\raggedleft\arraybackslash}p{(\linewidth - 14\tabcolsep) * \real{0.0972}}
  >{\raggedleft\arraybackslash}p{(\linewidth - 14\tabcolsep) * \real{0.0764}}
  >{\raggedleft\arraybackslash}p{(\linewidth - 14\tabcolsep) * \real{0.1319}}@{}}
\caption{Visão geral do labour share por região (2004--2025, a cada 5
anos).}\tabularnewline
\toprule\noalign{}
\begin{minipage}[b]{\linewidth}\raggedleft
year
\end{minipage} & \begin{minipage}[b]{\linewidth}\raggedleft
East Asia \& Pacific
\end{minipage} & \begin{minipage}[b]{\linewidth}\raggedleft
Europe \& Central Asia
\end{minipage} & \begin{minipage}[b]{\linewidth}\raggedleft
Latin America \& Caribbean
\end{minipage} & \begin{minipage}[b]{\linewidth}\raggedleft
Middle East \& North Africa
\end{minipage} & \begin{minipage}[b]{\linewidth}\raggedleft
North America
\end{minipage} & \begin{minipage}[b]{\linewidth}\raggedleft
South Asia
\end{minipage} & \begin{minipage}[b]{\linewidth}\raggedleft
Sub-Saharan Africa
\end{minipage} \\
\midrule\noalign{}
\endfirsthead
\toprule\noalign{}
\begin{minipage}[b]{\linewidth}\raggedleft
year
\end{minipage} & \begin{minipage}[b]{\linewidth}\raggedleft
East Asia \& Pacific
\end{minipage} & \begin{minipage}[b]{\linewidth}\raggedleft
Europe \& Central Asia
\end{minipage} & \begin{minipage}[b]{\linewidth}\raggedleft
Latin America \& Caribbean
\end{minipage} & \begin{minipage}[b]{\linewidth}\raggedleft
Middle East \& North Africa
\end{minipage} & \begin{minipage}[b]{\linewidth}\raggedleft
North America
\end{minipage} & \begin{minipage}[b]{\linewidth}\raggedleft
South Asia
\end{minipage} & \begin{minipage}[b]{\linewidth}\raggedleft
Sub-Saharan Africa
\end{minipage} \\
\midrule\noalign{}
\endhead
\bottomrule\noalign{}
\endlastfoot
2005 & 46.50462 & 51.90579 & 49.01374 & 34.62096 & 60.1775 & 47.44467 &
40.85213 \\
2010 & 46.75659 & 52.33425 & 49.57748 & 35.97861 & 59.7805 & 47.64883 &
40.73968 \\
2015 & 47.33338 & 51.57227 & 50.22565 & 41.33578 & 60.1975 & 46.72233 &
41.89534 \\
2020 & 47.96679 & 53.50062 & 49.95961 & 42.23030 & 60.4700 & 47.87667 &
41.81283 \\
2025 & 46.35841 & 52.11966 & 48.34071 & 38.25619 & 58.0670 & 45.85650 &
41.82416 \\
\end{longtable}

Conseguimos ver como os ``países de velho mundo'' são os únicos com um
\emph{labour share} acima dos 50\%. A nível global, isso não se
verifica. Embora que consiguamos registar um declínio da parte da
América do Norte de \(61\%\) em 2004, para \(58\%\) em 2025. A maioria
das regiões mostra certa estabilidade no valor (o que não significa ser
algo bom), tirando a América do Norte (com o declínio já mencionado), e
o Médio Oriente e Norte da África. O Médio Oriente e Norte da África,
Para além de terem o pior \emph{labour share} registado em (quase) todos
os anos (\(34\%\)--\(42\%\)), é bastante desregulado. Começou a subir
perto de 2007--2008, quanto teve a
\href{https://pt.wikipedia.org/wiki/Crise_financeira_de_2007\%E2\%80\%932008}{crise
global}. São países com grandes riquezas petrolíferas, e que quando o
preço do ``ouro negro'' subiu, o mesmo não se verificou na carteira dos
trabalhadores. A
\href{https://pt.wikipedia.org/wiki/Primavera_\%C3\%81rabe}{Primavera
Árabe} ocorreu após essa crise, e é provável que, para acalmar as
revoltas, os governos criaram reformas que aumentassem o \emph{labour
share} de certa forma. Também é possível notar o aumento a nível global
no ano 2020, com o evento do ínicio pandemia do
\href{https://pt.wikipedia.org/wiki/Coronav\%C3\%ADrus_da_s\%C3\%ADndrome_respirat\%C3\%B3ria_aguda_grave_2}{SARS-CoV-2}.

\includegraphics[width=1\linewidth]{analysis_files/figure-latex/labour-share-plot-1}

Uma coisa é certa, com um \emph{labour share} a nível global de cerca de
\(48\%\), e regiões onde o valor é \(10\%\) acima ou abaixo desse valor,
com tendências descendentes nas regiões de ambos os casos. Existe
claramente uma desigualdade regional, com os trabalhadores do norte
global a ganhar uma maior parte da riqueza que a sua região gera. Apesar
de declínios a nível regional, existe uma certa estabilidade global.

\pandocbounded{\includegraphics[keepaspectratio]{analysis_files/figure-latex/labour-share-heatmap-1.pdf}}

O \emph{heatmap} claramente destaca os mínimos das regiões do Médio
Oriente e Norte da África/Subsaariana em comparação com os máximos da
América do Norte.

\subsubsection{Comparação entre centro e
periferia}\label{comparauxe7uxe3o-entre-centro-e-periferia}

Conseguimos ver claramente uma diferença enorme entre os países com
maior rendimento e aqueles com menor rendimento. Existe uma diferença
maior que \(10\%\) sobre o \emph{labour share}. Os países de renda médio
alta e de renda médio baixa estão quase na mesma situação, e acima da
média dos extremos normalmente, com os países de renda média baixa a
passar para baixo nos últimos anos. Conseguimos reparar que os de renda
média baixa já tiveram em média maior \emph{labour share} que os de
renda médio alta até 2012. Mas também é de lembrar que talvez, os países
que hoje estão categorizados como sendo de renda média alta, antes de
2010 era de renda média baixa, e vice-versa, como já foi notado
anteriormente.

\includegraphics[width=1\linewidth]{analysis_files/figure-latex/dependency-theory-1}

Uma possível forma de demonstrar a
\href{https://pt.wikipedia.org/wiki/Teoria_da_depend\%C3\%AAncia}{teoria
da dependência} mostrando diferenças persistentes no labour share entre
economias centrais (alta renda) e periféricas (baixa renda), destacando
a dinâmica da exploração.

Os países centrais de rendimentos altos mantêm cerca de \(50\%\) do
labour share, enquanto os países periféricos de baixo rendimento
estagnam em cerca de \(38\%\), criando uma diferença estrutural de um
pouco mais de \(10\%\) que visualiza o mecanismo de ``dependência'' e
extração central da teoria da dependência.

\subsubsection{\texorpdfstring{Produtividade \emph{versus} \emph{labour
share}}{Produtividade versus labour share}}\label{produtividade-versus-labour-share}

Agora, vamos tentar mostrar uma contradição entre estas duas coisas:

\begin{itemize}
\tightlist
\item
  Produtividade (produção por trabalhador) --- medida em PIB por
  trabalhador (paridade do poder de compra em dólares), reflete o
  aumento da capacidade técnica da mão de obra e os ganhos de eficiência
  dentro do capitalismo. Isto captura diretamente a produção por
  trabalhador, tornando-o ideal para revelar a contradição entre o
  aumento da produtividade e a estagnação do \emph{labour share}. Como
  se concentra apenas nos trabalhadores empregados, evita distorções de
  fatores demográficos, como taxas de fertilidade ou reformados. O
  ajuste à paridade do poder de compra (PPC) garante a comparabilidade
  entre as diferentes regiões do globo;
\item
  Labor income share --- que representa a parte da riqueza efetivamente
  capturada pelos trabalhadores.
\end{itemize}

A produtividade tende a aumentar devido à inovação, mas \emph{labour
share} muitas vezes estagna ou diminui --- o que significa que o capital
captura mais da mais-valia.

A análise seguinte deve ajudar a tornar explícito esta contradição: os
trabalhadores aumentam cada vez mais a produtividade, enquanto a sua
reivindicação coletiva sobre o lucro (representado pelo \emph{labour
share}) é mediada e limitada pelo controlo do capital sobre a produção.

\includegraphics[width=1\linewidth]{analysis_files/figure-latex/productivity-labour-share-1}

Na maioria das regiões (excluindo o Médio Oriente e o Norte de África
até a altura do COVID), a produtividade só têm aumentado a nível global.
Enquanto isso, temos o \emph{labour share} basicamente estagnado.

\pandocbounded{\includegraphics[keepaspectratio]{analysis_files/figure-latex/mena-outlier-1.pdf}}

Se a relação fosse negativa ou estável, tornava explícito: à medida que
a produtividade cresce, a reivindicação dos trabalhadores não cresce.

Porém, afinal vemos que os ganhos de produtividade traduzem-se
parcialmente no \emph{labour share}, \textbf{mas apenas a nível global}.
E isso fica bem claro quando olhamos para a exceção: a região do Médio
Oriente e Norte de África, onde os rendimentos do petróleo, por estarem
sob controlo autoritário dos líderes da região, são eles que ficam com o
mais da mais-valia. A produtividade têm aumentado também, porém os
trabalhadores recebem cerca de \(35\%\)--\(40\%\), enquanto o capita
acumula mais de 60\%. O que isto mostra talvez na realidade é que esta é
a única região em que os capitalistas não aumentam, nem que ligeiramente
os salários, para manter os trabalhadores calmos, ao mesmo nível que do
resto do mundo. São dois modelos de extração do lucro diferentes: a
gerida e a descarada.

\begin{longtable}[]{@{}
  >{\raggedright\arraybackslash}p{(\linewidth - 8\tabcolsep) * \real{0.3913}}
  >{\raggedleft\arraybackslash}p{(\linewidth - 8\tabcolsep) * \real{0.1594}}
  >{\raggedleft\arraybackslash}p{(\linewidth - 8\tabcolsep) * \real{0.1449}}
  >{\raggedleft\arraybackslash}p{(\linewidth - 8\tabcolsep) * \real{0.1449}}
  >{\raggedleft\arraybackslash}p{(\linewidth - 8\tabcolsep) * \real{0.1594}}@{}}
\caption{Visão geral do labour share por região (2004--2025, a cada 5
anos).}\tabularnewline
\toprule\noalign{}
\begin{minipage}[b]{\linewidth}\raggedright
region
\end{minipage} & \begin{minipage}[b]{\linewidth}\raggedleft
slope
\end{minipage} & \begin{minipage}[b]{\linewidth}\raggedleft
std.error
\end{minipage} & \begin{minipage}[b]{\linewidth}\raggedleft
p.value
\end{minipage} & \begin{minipage}[b]{\linewidth}\raggedleft
statistic
\end{minipage} \\
\midrule\noalign{}
\endfirsthead
\toprule\noalign{}
\begin{minipage}[b]{\linewidth}\raggedright
region
\end{minipage} & \begin{minipage}[b]{\linewidth}\raggedleft
slope
\end{minipage} & \begin{minipage}[b]{\linewidth}\raggedleft
std.error
\end{minipage} & \begin{minipage}[b]{\linewidth}\raggedleft
p.value
\end{minipage} & \begin{minipage}[b]{\linewidth}\raggedleft
statistic
\end{minipage} \\
\midrule\noalign{}
\endhead
\bottomrule\noalign{}
\endlastfoot
South Asia & -0.0001262 & 0.0000310 & 0.0006620 & -4.0639675 \\
Sub-Saharan Africa & 0.0003493 & 0.0001454 & 0.0267061 & 2.4017787 \\
East Asia \& Pacific & -0.0000148 & 0.0000271 & 0.5912831 &
-0.5462030 \\
Middle East \& North Africa & -0.0006366 & 0.0000808 & 0.0000002 &
-7.8815888 \\
North America & -0.0001202 & 0.0000219 & 0.0000264 & -5.4987752 \\
Latin America \& Caribbean & -0.0000871 & 0.0000758 & 0.2652746 &
-1.1478594 \\
Europe \& Central Asia & -0.0000191 & 0.0000225 & 0.4047215 &
-0.8521926 \\
\end{longtable}

Quando olhamos \textbf{a nível regional}, a contradição mostra-se
correta (apenas a África Subsaariana contraria a tendência). A
``inclinação positiva'' global era uma ilusão ótica ---
\href{https://pt.wikipedia.org/wiki/Paradoxo_de_Simpson}{paradoxo de
Simpson} --- resultante da média de todas as regiões, mas regionalmente,
a tendência do capital é clara. Ao olharmos para o cenário globalmente,
acabamos por cair na narrativa do capital, que ganhos de produtividade
equivale a ganhos salariais. A análise regional expõe a mentira --- o
\emph{labour share} está basicamente estagnado em todos os lugares.

\includegraphics[width=1\linewidth]{analysis_files/figure-latex/unmasking-contradiction-1}

A linha global demonstra a ilusão de ótica resultante da mistura de
regiões alta produtividade com com baixa produtividade. O declínio do
\emph{labour share} é uma característica estrutural do capitalismo em
todo o mundo.

Vamos normalizar a produtividade numa distribuição normal para uma
última visualização.

\pandocbounded{\includegraphics[keepaspectratio]{analysis_files/figure-latex/focus-on-labour-share-1.pdf}}

\subsubsection{Investimento direto estrangeiro versus labour
share}\label{investimento-direto-estrangeiro-versus-labour-share}

Como os fluxos do IDE podem assumir valores negativos, aplicamos uma
transformação logarítmica com sinal, preservando tanto a magnitude como
a direção dos fluxos de capital, ao mesmo tempo que reduzimos a
assimetria.

Sem o uso da desta transformação, neste caso com a função logaritmica,
estariamos a juntar fluxos enormes com fluxos mais pequenos. Desta forma
os valores maiores ficam comprimidos, diminuindo a variação, continuam a
mostrar a sua dimensão, ao mesmo tempo que diminuimos os possíveis
\emph{outliers} que poderiam dificultar a análise do coeficiente de
correlação Pearson.

A transformação logarítmica com sinal preserva a direção do fluxo, ao
mesmo tempo que controla os valores monstruosos.

A correlação de Pearson usa a covariância que mede a direção do
movimento conjunto deste caso da média dos fluxos do IDE com a média do
\emph{labour share}, ao longo do tempo, e divide pela multipicação do
desvio padrão dos dois. O desvio padrão mede o quão dispersos estão os
dados em torno da sua média. Se estiverem muito dispersos, o valor é
maior, e isso vai então diminuir o valor do coeficiente de correlação.
No final, o coeficiente é um valor normalizado entre \(-1\) e \(1\). Se
o valor fôr \(0\), não existe correlação. Se fôr \(1\), existe uma
correlação prefeita onde um valor cresce juntamente com o outro, que é o
caso oposto de se o coeficiente fôr \(-1\), também uma correlação
prefeita, mas um dos valor desce quando o outro sobe.

Para a esta correlação entre os fluxos IDE e o \emph{labour share}, o
coeficiente \(< 0\) significa ``mais extração, menos \emph{labour
share}'' --- sinal imperialista fundamental.

Esta visualização destaca correlações negativas no Sul Global (América
Latina, Caraíbas, Médio Oriente e Norte de África) como evidência da
destruição salarial através da extração imperialistas, enquanto
correlações positivas em outros lugares expõem a colaboração entre
burguesias.

\pandocbounded{\includegraphics[keepaspectratio]{analysis_files/figure-latex/correlation-1.pdf}}

As correlações negativas na América Latina (\(-4,1\%\)) e na região
Médio Oriente e Norte de África (\(-10,4\%\)) dão força à tese de que os
fluxos IDE não representam ``desenvolvimento'', mas sim a extração de
valor dos trabalhadores para os centros imperialistas. O Leste Asiático
e o Pacífico apresentam a correlação positiva mais forte entre fluxos
IDE e \emph{labour share} (Ásia Oriental \(+17,7\%\)), sugerindo
resistência dos trabalhadores ou dinâmicas alternativas de capital como
alianças capitalistas estatais. A correlação positiva da Ásia Oriental
prova que a autogestão dos trabalhadores pode capturar/reverter os
fluxos imperialistas.

Os valores do coeficiente de correlação, podem não parecer
suficientemente fortes, até as regiões destacadas têm valores fracos.
Mas esta é uma análise empírica. As fracas correlações não negam a
extração imperialista. Correlações fracas são consistentes com um
sistema no qual a extração de excedentes está estruturalmente
incorporada.
